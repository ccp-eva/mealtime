% Options for packages loaded elsewhere
\PassOptionsToPackage{unicode}{hyperref}
\PassOptionsToPackage{hyphens}{url}
%
\documentclass[
  man,floatsintext]{apa6}
\usepackage{amsmath,amssymb}
\usepackage{iftex}
\ifPDFTeX
  \usepackage[T1]{fontenc}
  \usepackage[utf8]{inputenc}
  \usepackage{textcomp} % provide euro and other symbols
\else % if luatex or xetex
  \usepackage{unicode-math} % this also loads fontspec
  \defaultfontfeatures{Scale=MatchLowercase}
  \defaultfontfeatures[\rmfamily]{Ligatures=TeX,Scale=1}
\fi
\usepackage{lmodern}
\ifPDFTeX\else
  % xetex/luatex font selection
\fi
% Use upquote if available, for straight quotes in verbatim environments
\IfFileExists{upquote.sty}{\usepackage{upquote}}{}
\IfFileExists{microtype.sty}{% use microtype if available
  \usepackage[]{microtype}
  \UseMicrotypeSet[protrusion]{basicmath} % disable protrusion for tt fonts
}{}
\makeatletter
\@ifundefined{KOMAClassName}{% if non-KOMA class
  \IfFileExists{parskip.sty}{%
    \usepackage{parskip}
  }{% else
    \setlength{\parindent}{0pt}
    \setlength{\parskip}{6pt plus 2pt minus 1pt}}
}{% if KOMA class
  \KOMAoptions{parskip=half}}
\makeatother
\usepackage{xcolor}
\usepackage{graphicx}
\makeatletter
\def\maxwidth{\ifdim\Gin@nat@width>\linewidth\linewidth\else\Gin@nat@width\fi}
\def\maxheight{\ifdim\Gin@nat@height>\textheight\textheight\else\Gin@nat@height\fi}
\makeatother
% Scale images if necessary, so that they will not overflow the page
% margins by default, and it is still possible to overwrite the defaults
% using explicit options in \includegraphics[width, height, ...]{}
\setkeys{Gin}{width=\maxwidth,height=\maxheight,keepaspectratio}
% Set default figure placement to htbp
\makeatletter
\def\fps@figure{htbp}
\makeatother
\setlength{\emergencystretch}{3em} % prevent overfull lines
\providecommand{\tightlist}{%
  \setlength{\itemsep}{0pt}\setlength{\parskip}{0pt}}
\setcounter{secnumdepth}{-\maxdimen} % remove section numbering
% Make \paragraph and \subparagraph free-standing
\ifx\paragraph\undefined\else
  \let\oldparagraph\paragraph
  \renewcommand{\paragraph}[1]{\oldparagraph{#1}\mbox{}}
\fi
\ifx\subparagraph\undefined\else
  \let\oldsubparagraph\subparagraph
  \renewcommand{\subparagraph}[1]{\oldsubparagraph{#1}\mbox{}}
\fi
\newlength{\cslhangindent}
\setlength{\cslhangindent}{1.5em}
\newlength{\csllabelwidth}
\setlength{\csllabelwidth}{3em}
\newlength{\cslentryspacingunit} % times entry-spacing
\setlength{\cslentryspacingunit}{\parskip}
\newenvironment{CSLReferences}[2] % #1 hanging-ident, #2 entry spacing
 {% don't indent paragraphs
  \setlength{\parindent}{0pt}
  % turn on hanging indent if param 1 is 1
  \ifodd #1
  \let\oldpar\par
  \def\par{\hangindent=\cslhangindent\oldpar}
  \fi
  % set entry spacing
  \setlength{\parskip}{#2\cslentryspacingunit}
 }%
 {}
\usepackage{calc}
\newcommand{\CSLBlock}[1]{#1\hfill\break}
\newcommand{\CSLLeftMargin}[1]{\parbox[t]{\csllabelwidth}{#1}}
\newcommand{\CSLRightInline}[1]{\parbox[t]{\linewidth - \csllabelwidth}{#1}\break}
\newcommand{\CSLIndent}[1]{\hspace{\cslhangindent}#1}
\ifLuaTeX
\usepackage[bidi=basic]{babel}
\else
\usepackage[bidi=default]{babel}
\fi
\babelprovide[main,import]{english}
% get rid of language-specific shorthands (see #6817):
\let\LanguageShortHands\languageshorthands
\def\languageshorthands#1{}
% Manuscript styling
\usepackage{upgreek}
\captionsetup{font=singlespacing,justification=justified}

% Table formatting
\usepackage{longtable}
\usepackage{lscape}
% \usepackage[counterclockwise]{rotating}   % Landscape page setup for large tables
\usepackage{multirow}		% Table styling
\usepackage{tabularx}		% Control Column width
\usepackage[flushleft]{threeparttable}	% Allows for three part tables with a specified notes section
\usepackage{threeparttablex}            % Lets threeparttable work with longtable

% Create new environments so endfloat can handle them
% \newenvironment{ltable}
%   {\begin{landscape}\centering\begin{threeparttable}}
%   {\end{threeparttable}\end{landscape}}
\newenvironment{lltable}{\begin{landscape}\centering\begin{ThreePartTable}}{\end{ThreePartTable}\end{landscape}}

% Enables adjusting longtable caption width to table width
% Solution found at http://golatex.de/longtable-mit-caption-so-breit-wie-die-tabelle-t15767.html
\makeatletter
\newcommand\LastLTentrywidth{1em}
\newlength\longtablewidth
\setlength{\longtablewidth}{1in}
\newcommand{\getlongtablewidth}{\begingroup \ifcsname LT@\roman{LT@tables}\endcsname \global\longtablewidth=0pt \renewcommand{\LT@entry}[2]{\global\advance\longtablewidth by ##2\relax\gdef\LastLTentrywidth{##2}}\@nameuse{LT@\roman{LT@tables}} \fi \endgroup}

% \setlength{\parindent}{0.5in}
% \setlength{\parskip}{0pt plus 0pt minus 0pt}

% Overwrite redefinition of paragraph and subparagraph by the default LaTeX template
% See https://github.com/crsh/papaja/issues/292
\makeatletter
\renewcommand{\paragraph}{\@startsection{paragraph}{4}{\parindent}%
  {0\baselineskip \@plus 0.2ex \@minus 0.2ex}%
  {-1em}%
  {\normalfont\normalsize\bfseries\itshape\typesectitle}}

\renewcommand{\subparagraph}[1]{\@startsection{subparagraph}{5}{1em}%
  {0\baselineskip \@plus 0.2ex \@minus 0.2ex}%
  {-\z@\relax}%
  {\normalfont\normalsize\itshape\hspace{\parindent}{#1}\textit{\addperi}}{\relax}}
\makeatother

% \usepackage{etoolbox}
\makeatletter
\patchcmd{\HyOrg@maketitle}
  {\section{\normalfont\normalsize\abstractname}}
  {\section*{\normalfont\normalsize\abstractname}}
  {}{\typeout{Failed to patch abstract.}}
\patchcmd{\HyOrg@maketitle}
  {\section{\protect\normalfont{\@title}}}
  {\section*{\protect\normalfont{\@title}}}
  {}{\typeout{Failed to patch title.}}
\makeatother

\usepackage{xpatch}
\makeatletter
\xapptocmd\appendix
  {\xapptocmd\section
    {\addcontentsline{toc}{section}{\appendixname\ifoneappendix\else~\theappendix\fi\\: #1}}
    {}{\InnerPatchFailed}%
  }
{}{\PatchFailed}
\keywords{Language acquisition, Communication, Gesture, Cross-cultural psychology, Parent-child interaction\newline\indent Word count: 8454}
\usepackage{lineno}

\linenumbers
\usepackage{csquotes}
\ifLuaTeX
  \usepackage{selnolig}  % disable illegal ligatures
\fi
\IfFileExists{bookmark.sty}{\usepackage{bookmark}}{\usepackage{hyperref}}
\IfFileExists{xurl.sty}{\usepackage{xurl}}{} % add URL line breaks if available
\urlstyle{same}
\hypersetup{
  pdftitle={Mealtime conversations between parents and their 2-year-old children in five cultural contexts},
  pdfauthor={Manuel Bohn1,2, Wilson Filipe da Silva Vieira2, Marta Giner Torréns3, Joscha Kärtner3, Shoji Itakura4, Lília Cavalcante5, Daniel Haun2, Moritz Köster6,*, \& Patricia Kanngiesser7,*},
  pdflang={en-EN},
  pdfkeywords={Language acquisition, Communication, Gesture, Cross-cultural psychology, Parent-child interaction},
  hidelinks,
  pdfcreator={LaTeX via pandoc}}

\title{Mealtime conversations between parents and their 2-year-old children in five cultural contexts}
\author{Manuel Bohn\textsuperscript{1,2}, Wilson Filipe da Silva Vieira\textsuperscript{2}, Marta Giner Torréns\textsuperscript{3}, Joscha Kärtner\textsuperscript{3}, Shoji Itakura\textsuperscript{4}, Lília Cavalcante\textsuperscript{5}, Daniel Haun\textsuperscript{2}, Moritz Köster\textsuperscript{6,*}, \& Patricia Kanngiesser\textsuperscript{7,*}}
\date{}


\shorttitle{Mealtime conversations in five cultural contexts}

\authornote{

This research was funded by a grant from Volkswagen Foundation (89611--2), awarded to P.K.. The data assessment in Kyoto was partly funded by a JSPS Fellowship to M.K.. M.B. was supported by a Jacobs Foundation Research Fellowship. We would like to thank all the research assistants who supported the data assessments and coding: Phileas Heim, Chisato Fukuda, Julia Ohlendorf, Marlene Abromeit. We are grateful to Luke Maurits for statistical advice. All data and analysis code can be found in the following repository: \url{https://github.com/ccp-eva/mealtime}.

The authors made the following contributions. Manuel Bohn: Conceptualization, Methodology, Formal Analysis, Visualization, Writing -- original draft, Writing -- review \& editing; Wilson Filipe da Silva Vieira: Investigation, Writing -- review \& editing; Marta Giner Torréns: Investigation, Conceptualization, Methodology, Writing -- review \& editing; Joscha Kärtner: Investigation, Conceptualization, Methodology, Writing -- review \& editing; Shoji Itakura: Investigation, Writing -- review \& editing; Lília Cavalcante: Investigation, Writing -- review \& editing; Daniel Haun: Investigation, Writing -- review \& editing; Moritz Köster: Conceptualization, Methodology, Investigation, Writing -- review \& editing; Patricia Kanngiesser: Conceptualization, Methodology, Investigation, Writing -- review \& editing.

Correspondence concerning this article should be addressed to Manuel Bohn, Max Planck Institute for Evolutionary Anthropology, Deutscher Platz 6, 04103 Leipzig, Germany. E-mail: \href{mailto:manuel_bohn@eva.mpg.de}{\nolinkurl{manuel\_bohn@eva.mpg.de}}

}

\affiliation{\vspace{0.5cm}\textsuperscript{1} Institute of Psychology, Leuphana University Lüneburg, Lüneburg, Germany\\\textsuperscript{2} Department of Comparative Cultural Psychology, Max Planck Institute for Evolutionary Anthropology, Leipzig, Germany\\\textsuperscript{3} Department of Psychology, University of Münster, Münster, Germany\\\textsuperscript{4} Center for Baby Science, Doshisha University, Kyoto, Japan\\\textsuperscript{5} Graduate Program in Theory and Research of Behavior, Federal University of Pará, Belém, Brazil\\\textsuperscript{6} Institute of Psychology, University of Regensburg, Regensburg, Germany\\\textsuperscript{7} School of Psychology, University of Plymouth, Plymouth, UK\\\textsuperscript{*} joint senior authors}

\abstract{%
Children all over the world learn language, yet, the contexts in which they do so varies substantially. This variation needs to be systematically quantified to build robust and generalizable theories of language acquisition. We compared communicative interactions between parents and their two-year-old children (N = 99 families) during mealtime across five cultural settings (Brazil, Ecuador, Argentina, Germany, Japan) and coded the amount of talk and gestures as well as their conversational embedding (interlocutors, function, themes). We found a comparable pattern of communicative interactions across cultural settings, which were modified in ways that are consistent with local norms and values. These results suggest that children encounter similarly structured communicative environments across diverse cultural contexts and will inform theories of language learning.
}



\begin{document}
\maketitle

\hypertarget{public-significance-statement}{%
\section{Public significance statement}\label{public-significance-statement}}

Cultural norms and beliefs structure social interactions and communication. As a consequence, children learn language under very different circumstances. We studied communicative interactions between parents and their children in five diverse cultural contexts. We found a common, child-centered pattern of communication that was modified in line with local norms and values. This suggests that children can rely on similar information sources and learning processes across cultural contexts.

\hypertarget{introduction}{%
\section{Introduction}\label{introduction}}

Children learn language in interactions with language-competent others (Bohn \& Frank, 2019; Bruner, 1983; Clark, 2009; Levinson \& Holler, 2014; Tomasello, 2009). Social interactions between children and their social partners are structured by norms, values, and beliefs that vary substantially across cultural and historical contexts (Rogoff et al., 2003). As a consequence, children may encounter dramatically different language learning environments. Yet, the fact that children usually achieve fluency in their local language(s) suggests that they use a suite of compensatory learning strategies to adapt flexibly to their respective learning environment (Cristia, 2022; Kidd \& Garcia, 2022; Rowe \& Weisleder, 2020). Explaining how children accomplish this feat poses a serious theoretical and empirical challenge. Detailed documentation of learning environments across cultural contexts is needed to inform theorizing about children's learning processes. In this paper, we contribute to this effort by reporting on cross-cultural variation in parent-child communicative interactions in a semi-structured setting: meals involving parents and their 2-year-old child.

In recent decades, research on language acquisition has focused, to a large extent, on variation in language input and, in particular, the number of words children hear in naturalistic settings. This line of work was sparked by the finding that children who receive more input -- especially speech directly addressing them -- have larger vocabularies (Bang, Bohn, Ramirez, Marchman, \& Fernald, 2022; Hart \& Risley, 1995; Huttenlocher, Haight, Bryk, Seltzer, \& Lyons, 1991; Shneidman \& Goldin-Meadow, 2012; Walker, Greenwood, Hart, \& Carta, 1994; Weisleder \& Fernald, 2013). From a theoretical perspective, more language input increases children's opportunities for learning word-meaning mappings and allows them to build a larger vocabulary (Jones \& Rowland, 2017; Kachergis, Marchman, \& Frank, 2022; McMurray, Horst, \& Samuelson, 2012). The introduction of daylong audio recording devices and automated coding algorithms has provided further momentum to this endeavor (Cristia et al., 2021; Greenwood, Thiemann-Bourque, Walker, Buzhardt, \& Gilkerson, 2011; Lavechin, Bousbib, Bredin, Dupoux, \& Cristia, 2020). As a consequence, the quantity of direct language input plays a central role in theories and formal models of language learning (Braginsky, Yurovsky, Marchman, \& Frank, 2019; Goodman, Dale, \& Li, 2008; Kachergis et al., 2022; Swingley \& Humphrey, 2018).

However, like most of developmental psychology (Amir \& McAuliffe, 2020; Nielsen, Haun, Kärtner, \& Legare, 2017), research on language acquisition has largely focused on affluent societies of the Global North and the resulting theoretical proposals may fail to generalize to other cultural contexts. As studies in a greater variety of cultural settings have begun to accumulate (Altınkamış, Kern, \& Sofu, 2014; Bergelson et al., 2019; Bunce et al., 2020; Casillas, Brown, \& Levinson, 2021; Choi, 2000; Cristia, Dupoux, Gurven, \& Stieglitz, 2019; Loukatou, Scaff, Demuth, Cristia, \& Havron, 2021; Tardif, Shatz, \& Naigles, 1997), they have revealed substantial cultural variation in how much direct input children receive (Cristia, 2022; see also Sperry, Sperry, \& Miller, 2019 for variation within an English-speaking sample). Yet, children still reach major milestones in language development at similar ages (Brown \& Gaskins, 2014; Casillas, Brown, \& Levinson, 2020). These findings highlight that theories and models of language learning need to extend beyond quantity of input and also include learning processes that compensate for variation in input (Bang, Mora, Munévar, Fernald, \& Marchman, 2022; Casillas, 2022; Jones \& Rowland, 2017; Kachergis et al., 2022; Meylan \& Bergelson, 2022).

It has been suggested that these compensatory learning processes leverage structural features of social interactions in which language is used (Casillas et al., 2020; Rogoff, Paradise, Arauz, Correa-Chávez, \& Angelillo, 2003; Shneidman \& Goldin-Meadow, 2012; Shneidman \& Woodward, 2016). Pragmatic accounts of language learning offer an explanation for how children use contextual information (e.g., Bohn \& Frank, 2019; Tomasello, 2009): Social interactions, especially routines, follow predictable patterns that make it easier for children to infer what speakers are communicating about (Barbaro \& Fausey, 2022; Bruner, 1983; Lieven, 1994; Masek, Ramirez, McMillan, Hirsh-Pasek, \& Golinkoff, 2021; Vygotsky, 1978). For instance, Roy, Frank, DeCamp, Miller, and Roy (2015) found that words were more easily learned when they were primarily used in a distinct spatial and temporal context. Similarly, establishing common ground over the course of an interaction provides information about the speaker's intention independent of the words that are being used (Bohn \& Köymen, 2018; Bohn, Tessler, Merrick, \& Frank, 2021). For example, Bohn, Le, Peloquin, Köymen, and Frank (2021) showed that children identify the referent of an ambiguous word by inferring the topic of an ongoing conversation (see also Akhtar, 2002). These findings help to explain why the amount of conversational turn-taking in parent-child interactions predicts child language outcomes (Donnelly \& Kidd, 2021; Romeo et al., 2018). Turn-taking results in continuous, structured conversations that provide information-rich learning opportunities.

To assess whether children can use structural features to complement direct verbal input, it is crucial to compare communicative interactions between adults and children across cultural settings. However, to our knowledge, there are very few quantitative comparisons. While ethnographic descriptions offer important and rich insights into individual cultural settings (see e.g., De León, 2011; Gaskins, 2006), quantitative comparisons are essential for understanding gradual cultural differences (Broesch et al., 2021; Hewlett, Lamb, Shannon, Leyendecker, \& Schölmerich, 1998; Köster et al., 2022) and offer core input for theory building.

One of the challenges of cross-cultural work lies in selecting an appropriate context for comparing the structure of communicative interactions (Broesch, Lew-Levy, Kärtner, Kanngiesser, \& Kline, 2022). Prior work has shown that the amount of language input children receive varies substantially across routine activities. For example, Soderstrom and Wittebolle (2013) found that Canadian adults spoke most during book reading and structured playtime (see also Tamis-LeMonda, Custode, Kuchirko, Escobar, \& Lo, 2019). Such activities, however, are very specific to industrialized societies and less frequent or absent in other cultural contexts. A cross-culturally recurrent, and hence particularly promising, context for cross-cultural research are mealtimes: across societies, meals are social events that are structured by -- and used to transmit -- cultural norms, values and beliefs (Blum-Kulka, 2012; Fjellström, 2004; Köster et al., 2022; Ochs \& Shohet, 2006). Furthermore, mealtimes have proven fruitful for studying caregiver-child communication in cultural contexts like the U.S. (e.g., Beals, 1993, 1997; Snow \& Beals, 2006).

\hypertarget{the-current-study}{%
\section{The current study}\label{the-current-study}}

The goal of this study was to compare communicative interactions between parents and their children during mealtimes across diverse cultural settings. We aimed for a naturalistic but comparable setup by a) asking families to record in their homes, b) recruiting families with a single -- usually the first -- child between 2 and 3 years of age and c) focusing on 10-minute-long episodes during which three family members (mother, father, one child) were present. Even though the constellation of two parents and one child might be less representative of the overall family demographics in some settings, it allowed us to directly quantify and compare communicative interactions.

We obtained recordings from five different cultural settings, including families living in the Global South and the Global North, as well as in urban and rural settings: the city of Buenos Aires in Argentina, small villages in the Amazon region near Apeú in Brazil), small villages close to Cotacachi in Ecuador, the city of Münster in Germany, and the city of Kyoto in Japan. This sample was first and foremost a convenience sample of families in diverse cultural settings we had worked with previously. This continues to be a common approach in larger-scale cross-cultural, developmental studies (P. R. Blake et al., 2015; House et al., 2020, 2013; Kanngiesser et al., 2022) and is often the first step when little substantive cross-cultural data exists to inform targeted comparisons. Nevertheless, in addition to their geographic spread and variation in population density, the settings also varied in cultural norms and beliefs about communication during mealtimes. In Germany, meals are seen as a privileged time for communication and exchange (Danesi, 2018). Similarly, in Argentina, dinners are an important opportunity for family conversations because it is usually the only time when the whole family gets together (Aguirre, 2016). In contrast, within the Kichwa indigenous people in Ecuador meals are supposed to be taken in silence (Sánchez-Parga, 2010). In Japan, both views are common and whether or not talk is encouraged depends, in part, on the eating arrangements (Imada \& Furumitsu, 2020). As such, our sample provided us with the opportunity to study if and how different cultural mealtime norms impact real-world communicative interactions.

We coded and analyzed our video data along several dimensions, focusing on the quantity of talk and gestures as well as their conversational embedding . We chose dimensions that have been implicated as relevant for child language acquisition, but have rarely been studied from a cross-cultural perspective. First, we coded the presence (or absence) of speech, the identity of the speaker and the recipient. This allowed us to quantify how much directed talk -- as opposed to overheard talk -- children received and from whom. As noted above, cross-cultural variation in talk directed at the child has profound theoretical implications because it questions the privileged role given to direct input in many theoretical accounts of language learning. Coding speaker identity provided insight into who children receive language input from. Cross-cultural research on different sources of language input is relatively scarce: most past studies have exclusively focused on maternal talk and only recently have researchers begun to investigate paternal talk (Ferjan Ramírez, 2022). By coding the language produced by children themselves, we were able to quantify children's role in shaping their linguistic environment across cultural contexts (Donnellan, Bannard, McGillion, Slocombe, \& Matthews, 2020; Tamis-LeMonda, Kuchirko, \& Suh, 2018). In addition to speech, we also coded the production of gestures. A substantial body of research has shown that gestures produced by children and their caregivers relate to child language competency -- at least in children growing up in the Global North (Colonnesi, Stams, Koster, \& Noom, 2010; e.g., Rowe, Özçalışkan, \& Goldin-Meadow, 2008). Here, the view is that gestures act as a complementary source of input that reference objects and events in the environment and thereby facilitate word learning (Tomasello, 2005).

Second, we coded how utterances were grouped into themes. This approach allowed us to quantify cross-cultural variation in how conversations are structured. Research on conversational turn-taking has suggested a link between these structural features and language learning (Donnelly \& Kidd, 2021; Romeo et al., 2018); yet, a cross cultural perspective is still largely missing. Finally, we coded the function of utterances and distinguished between questions, assertions and imperatives. Questions play a role in facilitating language acquisition because they encourage verbal responses from children which may include labels for objects (J. Blake, Macdonald, Bayrami, Agosta, \& Milian, 2006). There is also suggestive evidence of cultural variation in how parents use functional elements of language such as questions (Kuchirko, Schatz, Fletcher, \& Tamis-Lemonda, 2020).

For the analysis, we first assessed if and how these coded dimensions differed in the five cultural settings. In a second step, we asked whether some cultural settings are more similar to one another. The five cultural settings offer an interesting perspective on the factors influencing mealtime conversations. For example, communicative interaction patterns could cluster by country (five clusters; one cluster per country), or by language family and geographical region (three clusters; Argentina, Brazil, Ecuador vs.~Germany vs.~Japan) or by degree of urbanization (two clusters; urban: Argentina, Germany, Japan vs.~rural: Brazil, Ecuador). Based on previous work, we expected less direct input to children in the rural contexts (Cristia, 2022). Due to different cultural norms around mealtime conversations, we predicted less overall talk in Ecuador compared to Germany, with Japan falling somewhere in the middle. Given a lack of comparable previous work -- we had no specific predictions for variation in the structure of communicative interactions.

\hypertarget{methods}{%
\section{Methods}\label{methods}}

\hypertarget{transparency-and-openness.}{%
\subsection{Transparency and openness.}\label{transparency-and-openness.}}

We report how we determined our sample size, all data exclusions, all manipulations, and all measures in the study. All data and analysis code can be found in the following repository: \url{https://github.com/ccp-eva/mealtime}. Data were analyzed using R, version 4.2.0 (R Core Team, 2022) and the function \texttt{brm} from the package \texttt{brms} (Bürkner, 2017). We used default priors built into \texttt{brms} for all parameters. The study's design and its analysis were not pre-registered.

\hypertarget{participants}{%
\subsection{Participants}\label{participants}}

The final sample consisted of 99 families from five cultural contexts. This included 20 families from the city of Buenos Aires, Argentina (urban setting), 18 families from villages in the Amazon region near Apeú, Brazil (rural setting), 13 from villages near Cotacachi, Ecuador (rural setting), 24 families from the city of Münster, Germany (urban setting) and 24 families from the city of Kyoto, Japan (urban setting). For the recording sessions, all families comprised a father, a mother and a child aged between 2 years and 3 years, 2 months. Almost all children were the first child in the family. Some videos partly included additional children (n = 1 for Argentina, Brazil and Ecuador, respectively).

Additional families were recorded but they did not meet the inclusion criteria of at least one recording of a meal that lasted for at least ten minutes, initially included all three family members and had all family members visible in the recording. This resulted in the exclusion of 11 families from Münster, Germany, 34 from Apeú, Brazil, five from Buenos Aires, Argentina, 39 from Cotacachi, Ecuador and five from Kyoto, Japan.

The recordings were collected as part of a larger cross-cultural investigation into parent-child interactions and findings on parental teaching behaviors have been published by Köster et al. (2022). We refer to this earlier work for a detailed description of each cultural setting. In the following we only provide a short overview.

\hypertarget{argentina}{%
\subsubsection{Argentina}\label{argentina}}

Families lived in the metropolitan area of Buenos Aires, Argentina, which comprises around 15.2 million people. They were recruited via personal contacts of the local experimenter. The family language was Rioplatense Spanish. Compensation included small toys for children and USD 10 for parents. Most parents had completed a university degree (mothers: 74\%; fathers: 52\%) and engaged in paid professional labor (mothers: 87\%; fathers: 78\%). The majority of children (91\%) either attended kindergarten or were looked after by a nanny or a family member other than the parents.

\hypertarget{brazil}{%
\subsubsection{Brazil}\label{brazil}}

Families lived in villages of around 50 - 300 families in the Amazon region near Apeú, approximately 1.5 hours east of Belém, the capital of the state of Pará. They were recruited with the help of a local public health office. The family language was Brazilian Portuguese. Compensation included small toys for children and a certificate of participation for parents. Most parents had completed secondary school (\textasciitilde12 years of schooling, mothers: 50\%; fathers: 56\%). Mothers worked mainly as housewives (83\%) while fathers engaged in paid labor (100\%). Some families engaged in traditional subsistence activities such as tapioca farming, livestock breeding, or açaí and fruit harvesting. In line with employment status, the majority of children were looked after by their mothers.

\hypertarget{ecuador}{%
\subsubsection{Ecuador}\label{ecuador}}

Families identified as belonging to the Kichwa community and lived in villages with 800-5,000 inhabitants located within 1 hour (by car) of the city of Cotacachi in the Imbabura province. They were recruited via personal contacts mediated by the community president. The family language was Ecuadorian Spanish with elements of Kichwa. Compensation included food (e.g., rice or oat) and USD 4. Most parents had completed primary school (\textasciitilde10 years of schooling, mothers: 50\%; fathers: 56\%). Mothers worked mainly as housewives (59\%) while fathers engaged in paid labor (77\%). Around 40\% of children were looked after by a person other than the mother during the day.

\hypertarget{germany}{%
\subsubsection{Germany}\label{germany}}

Families lived in Münster in the state of North-Rhine-Westphalia, a city with \textasciitilde310,000 inhabitants. They were recruited via a participant database of the Developmental Psychology lab at the University of Münster. Compensation included a voucher of EUR 15 for a local toy store. Most parents had completed a university degree (mothers: 71\%; fathers: 71\%) and engaged in paid professional labor (mothers: 92\%; fathers: 92\%). All children either attended kindergarten or were looked after by a nanny during the day.

\hypertarget{japan}{%
\subsubsection{Japan}\label{japan}}

Families lived in the city of Kyoto, in the Kansai metropolitan region, with around 1.5 million inhabitants. They were recruited via a participant database of the Center for Baby Science at Doshisha University. Compensation was JPY 3000. Most parents had completed a university degree (mothers: 92\%; fathers: 83\%) and engaged in paid professional labor (mothers: 71\%; fathers: 100\%). Most children (80\%) attended kindergarten.

The study was approved by the ethics committee of the Free University of Berlin. Recordings took place between September 2017 and March 2019. Informed verbal consent was obtained from both parents and written consent from one of the parents.

\hypertarget{procedure}{%
\subsection{Procedure}\label{procedure}}

We visited families twice. On the first visit, an experimenter (familiar with the local language) instructed parents on how to use the video camera and what to record. We encouraged families to record two instances of the meal they commonly shared together, which happened in the evening for most families. The cameras were equipped with a wide-angle lens and set up to capture all family members during the meal. In addition to video, the cameras also recorded sound. On the second visit, the experimenter asked about the recordings and encouraged families to record additional meals if they had not already recorded two sessions. In the end, we collected socio-demographic information and interviewed the mothers (unrelated to the present study).

\hypertarget{coding}{%
\subsection{Coding}\label{coding}}

We scanned all recordings for sections that captured a meal event, lasted at least 10 minutes, and included all three family members. For each family, we selected one such section for in-depth coding and excluded all families for which we did not find such a section (see above for the number of excluded families).

We coded videos using ELAN (Wittenburg, Brugman, Russel, Klassman, \& Sloetjes, 2006) version 6.4. The primary coder was either a native (Germany, Japan, Brazil) or a highly fluent (Argentina, Ecuador) speaker of the local language. For Ecuador, a native speaker translated sections containing Kichwa into Spanish before the primary coder coded them.

In a first pass, the primary coder created a tier for each speaker and marked segments in which this person was speaking or using a gesture. In a second pass, the coder transcribed all utterances into the local language and coded their conversational embedding. We defined utterances as sections of continuous talk by one person. If speakers paused for more than 2 seconds, we coded two utterances with 2 (or more) seconds of silence in between. We used the following codes to capture the conversational embedding of each utterance:

\hypertarget{speaker}{%
\subsubsection{Speaker}\label{speaker}}

Here we coded who produced the utterance. The speaker could either be \texttt{child}, \texttt{mother}, or \texttt{father}. All sections containing no speech were coded as \texttt{non-talk}.

\hypertarget{recipient}{%
\subsubsection{Recipient}\label{recipient}}

Here we coded who the utterance was addressed to. Codes could either be \texttt{child}, \texttt{mother}, \texttt{father}, \texttt{both} or \texttt{other}, where \texttt{other} was used either when a fourth person (e.g., over the phone) was addressed or the speaker was talking to themselves (e.g., child babbling or singing). If an utterance addressed two people in sequence, the second addressee was coded as the recipient.

\hypertarget{themes-and-utterances}{%
\subsubsection{Themes and utterances}\label{themes-and-utterances}}

Here we coded the conversational coherence of the different utterances. For that we defined \texttt{themes} as sequences of utterances that related to one another. This applies for example to sequences of questions and answers but also to sequences in which the content of an utterance is directly related to the content of the previous utterance. Please note that such themes were coded locally and were not the same as topics. For example, if father and child exchanged four utterances about the child's day in the kindergarten this was coded as one theme. If the same topic (day at the kindergarten) came up later again, this was coded as a separate theme. Each utterance within a theme was counted to capture the sequence and length of a theme. Thus, each utterance was assigned a number for the theme and a number for the utterance within the theme. Themes could have interjections of one or two utterances. After more than two interjections we coded a new theme. For example, if father and child talked about food and the mother made an unrelated comment in between, the mother's comment would be coded as a separate theme while the other theme continued around it:

Child: ``I want more'' (theme (t) 1, utterance (u) 1)

Father: ``Do you want more soup?'' (t1, u2)

Mother: ``Phew, I'm hot (t2, u1)

Child: ``No, bread (t1, u3)

Father: ``I'll get some'' (t1, u4)

\hypertarget{functional-elements}{%
\subsubsection{Functional elements}\label{functional-elements}}

Each utterance was coded as either being a \texttt{question}, \texttt{assertion} or \texttt{imperative}. Imperatives were only coded if the utterance was grammatically structured as an imperative. For example ``Pass me the salt!'' was coded as an imperative while ``You should give me the salt.'' was not.

\hypertarget{referential-gestures}{%
\subsubsection{Referential gestures}\label{referential-gestures}}

We also coded the frequency of two types of referential gestures for each individual. \texttt{Points} were coded when someone indicated an object, location or person in the environment, either using a finger (often index finger), the head or an object (e.g., cutlery). Reaches and hold-outs were not coded as points. \texttt{Iconic} gestures were coded when someone depicted an object or action using their hands and/or body (e.g., pretending to hold a knife and cut to instruct the child how to cut a cucumber). Conventional gestures such as head shaking, nodding or shrugging were not coded.

\hypertarget{reliability-coding}{%
\subsubsection{Reliability coding}\label{reliability-coding}}

For each cultural setting, we selected 15\% of videos and had them re-coded by a second coder (native speaker of the respective language). The second coder relied on the sequencing of the primary coder. Inter-rater reliability was generally very good. For recipient, the agreement between coders was 88\% (\(\kappa\) = 0.83), for function it was 91\% (\(\kappa\) = 0.78) and for gestures it was 96\% (\(\kappa\) = 0.81). To get inter-rater reliability for the coding of themes, we asked whether the two coders agreed on whether a given utterance belonged to the same theme as the previous utterance or belonged to a new theme. Once again, agreement between coders was high (agreement = 87\%, \(\kappa\) = 0.74).

\hypertarget{analysis-and-results}{%
\section{Analysis and Results}\label{analysis-and-results}}

For each of the research questions (see below), we defined a response variable and then used Bayesian multilevel regression models to model the effect of cultural setting and -- whenever applicable -- that of the different individuals involved in the conversation. To make inferences about the importance of predictors, we compared a set of nested models including cultural setting and individual as predictors to each other and to a null model that did not include them to test if these predictors improved model fit. Following McElreath (2018), we compared models using Widely Applicable Information Criteria (WAIC). This approach favors models that have high out-of-sample predictive accuracy in that they achieve a good fit to the data with the minimal set of parameters.

We modeled the effect of cultural settings as random effects and interactions between additional variables (e.g., speaker identity) and setting as random slopes within cultural setting (\texttt{brms} notation: \texttt{(variable\textbar{}setting)}). This approach partially pools model estimates and is thought to yield more generalizable results because it avoids overfitting the model to the observed data (Gelman \& Hill, 2006; McElreath, 2018). For each model comparison, we report the difference in WAIC estimates, the standard error of the difference and the weight of each model. Model weights give the probability that a model will make the best predictions out of all the models considered.

Each model comparison has a ``winning'' model, that is, a model that has the lowest WAIC value and the highest weight and thus, the highest expected out-of-sample predictive accuracy. However, two models can be more or less equivalent when the difference in WAIC is small and the standard error of the difference is larger than the difference in WAIC. In addition to the model comparison, we visualize the predictions of the winning model and interpret them based on their posterior means and 95\% Credible Intervals (CrI).

\hypertarget{how-much-time-did-families-spend-talking}{%
\subsection{How much time did families spend talking?}\label{how-much-time-did-families-spend-talking}}

First, we ask how much time families spent talking as opposed to not talking and how this varied across cultural settings. The dependent variable in this case was the total lengths of all sections coded as non-talk for each family (modeled as a normal distribution). We compared a null model including only an overall intercept (\texttt{non-talk\ \textasciitilde{}\ 1}) to a model including cultural setting (\texttt{non-talk\ \textasciitilde{}\ 1\ +\ (1\textbar{}setting)})\footnote{One might suspect that the child's age influences their own behavior or that of the parents. To explore this possibility, we added models including child age as a predictor to the model comparison for the first three models (overall talk, talk per speaker and talk received by each individual). The inclusion of age did not improve the fit of the otherwise best fitting model any further. In the interest of space and readability, we do not report these models here. However, age is included in the data set available in the associated repository so interested readers can further explore the relation between age and the variables we coded.}.

The model comparison clearly favored the model including cultural setting (WAIC = 338.84, se = 14.93, weight \textgreater{} 0.99) over the null model (WAIC = 362.36, se = 14.97, weight \textless{} 0.01). The difference in WAIC (dWAIC) was = -11.76 with a standard error of 4.15. The model predicted an average of 4.95 {[}95\%CrI = 3.80 - 6.07{]} minutes of non-talk across cultural settings. Ecuador and Brazil had longer sections of non-talk compared to Argentina and Germany, with Japan falling in the middle (see Figure \ref{fig:fig1}A).

\hypertarget{how-much-talk-is-directed-at-each-family-member}{%
\subsection{How much talk is directed at each family member?}\label{how-much-talk-is-directed-at-each-family-member}}

Next, we asked whom the talk was directed to, that is, how much ``input'' each family member received. The dependent variable was the total lengths of utterances directed at each individual in a family. This variable was right-skewed and we therefore modeled it as a skewed normal distribution. Given that the analysis above showed that the amount of overall talk differed across cultural settings, the null model already included a random effect for setting (\texttt{input\ \textasciitilde{}\ 1\ +\ (1\textbar{}setting)\ +\ (1\textbar{}family)}). We compared it to two alternative models, one assuming that input additionally differed across recipients (\texttt{input\ \textasciitilde{}\ recipient\ +\ (1\textbar{}setting)\ +\ (1\textbar{}family)}) and one assuming that this effect in turn varies across settings (\texttt{input\ \textasciitilde{}\ recipient\ +\ (recipient\textbar{}setting)\ +\ (1\textbar{}family)}).

The model comparison clearly favored the two alternative models, with a slight preference for the simpler model that did not assume the effect of recipients to vary across cultural setting (WAIC = 705.72, se = 30.16, weight = 0.74; model assuming variation across settings: WAIC = 707.82, se = 30.15, weight = 0.26; dWAIC = -1.05, SE(dWAIC) = 0.85). We observed that, across settings, more talk was directed at children compared to the two parents with fathers being talked to the least (see Figure \ref{fig:fig1}B).

\hypertarget{which-family-member-talks-the-most}{%
\subsection{Which family member talks the most?}\label{which-family-member-talks-the-most}}

In the next analysis, we asked how talking time was distributed across the different family members. The dependent variable was the total lengths of utterances of each individual in a family, which was also right-skewed and modeled as a skewed normal distribution. Given previous results, the null model included a random effect for setting (\texttt{talk\ \textasciitilde{}\ 1\ +\ (1\textbar{}setting)\ +\ (1\textbar{}family)}). The first alternative model assumed that talk differed across speakers (\texttt{talk\ \textasciitilde{}\ recipient\ +\ (1\textbar{}setting)\ +\ (1\textbar{}family)}), the second assumed that this effect interacted with setting (\texttt{talk\ \textasciitilde{}\ recipient\ +\ (recipient\textbar{}setting)\ +\ (1\textbar{}family)}).

The model comparison clearly favored the interaction model assuming that the the difference between speakers varied across settings (WAIC = 755.92, se = 25.20, weight \textgreater{} 0.99; model assuming no interaction: WAIC = 772.14, se = 24.65, weight \textless{} 0.01; dWAIC = -8.11, SE(dWAIC) = 3.65). Figure \ref{fig:fig1}C shows that even though mothers talked the most in all settings, this effect was much more pronounced in Japan, Germany and Argentina compared to Ecuador and Brazil.

\hypertarget{how-many-gestures-are-being-used}{%
\subsection{How many gestures are being used?}\label{how-many-gestures-are-being-used}}

To conclude the first set of analysis, we looked at variation in gesture production. Iconic gestures were produced at a much lower rate (15\%) compared to pointing gestures (85\%). Thus, many individuals from different cultural settings did not produce any iconic gestures. This made it difficult to analyze points and iconic gestures separately and we instead decided to combine them. Thus, the dependent variable was the number of gestures produced by each individual. We modeled this distribution as a zero-inflated poisson distribution to account for the fact that some individuals did not produce any gestures.

The null model only included an intercept and a random effect of family (\texttt{gestures\ \textasciitilde{}\ 1\ +\ (1\textbar{}family)}). There were three alternative models: the first included producer (child, mother, father) as a fixed effect (\texttt{gestures\ \textasciitilde{}\ producer\ +\ (1\textbar{}family)}), the second model added to this a random effect for setting (\texttt{gestures\ \textasciitilde{}\ producer\ +\ (1\textbar{}setting)\ +\ (1\textbar{}family)}) and the third model included an additional random slope for interlocutors within setting to model the interaction (\texttt{gestures\ \textasciitilde{}\ producer\ +\ (producer\textbar{}setting)\ +\ (1\textbar{}family)}).

The model comparison clearly favored the model assuming that the number of gestures produced varied between individuals within cultural settings (interaction model; WAIC = 1602.79, se = 49.79, weight \textgreater{} 0.99; second best model (without interaction): WAIC = 1670.90, se = 53.44, weight \textless{} 0.01; dWAIC = -34.06, SE(dWAIC) = 14.33). Overall, there were slightly fewer gestures in Ecuador and Brazil. Looking at the different individuals, we saw that -- across settings -- children produced the most gestures, followed by mothers and then fathers. This pattern was less pronounced in Brazil and Argentina and notably reversed in Ecuador, where children produced hardly any gestures (see Figure \ref{fig:fig1}D).

\begin{figure}
\includegraphics[width=1\linewidth]{../visuals/fig1} \caption{A: Non-talk across cultural settings. B: Talk directed at the different individuals. C: Time spent talking by the different individuals. D: Number of gestures (points and iconic gestures combined) produced by each individual. In B-D: color denotes the individual. Distributions show the predicted values based on the respective model with solid points and error bars showing the mean with 66\% and 95\% CrI. Light points show the aggregated data for each family and -- whenever applicable -- individual.}\label{fig:fig1}
\end{figure}

\hypertarget{who-talks-to-whom}{%
\subsection{Who talks to whom?}\label{who-talks-to-whom}}

To address the question of who talks to whom we categorized the conversational partners of each utterance as either being mother and father, child and mother or child and father. We then used a categorical model to predict the proportion with which each of these categories occurred. The null model only included an intercept and a random effect of family (\texttt{partners\ \textasciitilde{}\ 1\ +\ (1\textbar{}family)}) while the alternative model assumed that these proportions differ across settings (\texttt{partners\ \textasciitilde{}\ 1\ +\ (1\textbar{}setting)\ +\ (1\textbar{}family)}).

The model comparison yielded no clear difference between models, suggesting no substantial differences in the proportion of conversational partners across settings (alternative model: WAIC = 28106.33, se = 116.90, weight = 0.62; null model: WAIC = 28107.31, se = 116.83, weight = 0.38; dWAIC = -0.49, SE(dWAIC) = 0.80). Compared to an equal split (proportion of 0.33 for each category), conversations between mother and child were slightly more frequent and conversations between child and father less frequent except for Brazil where conversations between mother and father were less likely (see Figure \ref{fig:fig2}A).

\hypertarget{who-uses-which-functional-elements}{%
\subsection{Who uses which functional elements?}\label{who-uses-which-functional-elements}}

As the next step, we analyzed how the different speakers used different functional elements -- assertions, imperatives, and questions. That is, we predicted the proportion with which each functional element occurred using a categorical model. We investigated whether the types of functional elements used varied with speakers as well as cultural settings. The null model only included an intercept and a random effect of family (\texttt{function\ \textasciitilde{}\ 1\ +\ (1\textbar{}family)}). There were three alternative models: the first included speaker as an additional fixed effect (\texttt{function\ \textasciitilde{}\ speaker\ +\ (1\textbar{}family)}), the second model added to this a random effect for setting (\texttt{function\ \textasciitilde{}\ speaker\ +\ (1\textbar{}setting)\ +\ (1\textbar{}family)}) and the third model included and additional random slope for speaker within setting to model the interaction between speaker and setting (\texttt{function\ \textasciitilde{}\ speaker\ +\ (speaker\textbar{}setting)\ +\ (1\textbar{}family)}).

The model comparison clearly favored the interaction model assuming that the use of functional element varied across speakers within cultural setting (WAIC = 23591.46, se = 180.20, weight \textgreater{} 0.99; second best model (without interaction): WAIC = 23689.02, se = 181.03, weight \textless{} 0.01; dWAIC = -48.78, SE(dWAIC) = 10.42). The general pattern was that assertions were the most frequent type of functional element, followed by questions and imperatives. This ordering was much more pronounced in children in that they hardly used questions or imperatives. Variation across settings was most notable in that both mothers and fathers from Brazil and Ecuador were substantially more likely to use imperatives compared to the other three settings (see Figure \ref{fig:fig2}B).

\begin{figure}
\includegraphics[width=1\linewidth]{../visuals/fig2} \caption{A: Proportion of utterances that were exchanged by a pair of interlocutors. Color shows the interlocutors involved in the utterance regardless of direction (i.e., identity of speaker and listener). B:Proportion of utterances that belonged to a certain class of functional element. Facets show different speakers, color denotes the functional element. Distributions show the predicted values based on the respective model with solid points and error bars showing the mean with 66\% and 95\% CrI. Light points show the aggregated data for each family.}\label{fig:fig2}
\end{figure}

\hypertarget{how-many-people-are-involved-in-a-theme}{%
\subsection{How many people are involved in a theme?}\label{how-many-people-are-involved-in-a-theme}}

Next, we turned to themes as the focus of analysis. As a first step, we asked how many different speakers were involved in a theme. To be involved in a theme, an individual had to produce at least one utterance. Please note that it was possible for themes to have only one speaker. In fact, this was the case for 34\% of all utterances. These themes were mostly single utterances that occurred when someone made an unrelated comment or asked a question but did not receive an answer. We counted the number of speakers involved in each theme (1, 2, or 3) and modeled the resulting distribution using a binomial model. Note that this approach does not take into account the length of each theme. We compared a null model including only an overall intercept (\texttt{no\_speakers\ \textasciitilde{}\ 1}) to a model including cultural setting (\texttt{no\_speakers\ \textasciitilde{}\ 1\ +\ (1\textbar{}setting)}).

The model comparison favored the model including cultural setting (WAIC = 6544.58, se = 41.13, weight = 0.95) over the null model (WAIC = 6550.36, se = 40.88, weight = 0.05; ; dWAIC = -2.89, SE(dWAIC) = 2.77). Figure \ref{fig:fig3}A shows that the number of speakers involved in a theme was relatively similar across cultural settings, with Brazil being the notable exception in having, on average, more speakers per theme.

\hypertarget{who-initiates-themes}{%
\subsection{Who initiates themes?}\label{who-initiates-themes}}

In the following analysis, we asked whether there are differences among speakers and cultural settings in who initiated a theme. For each theme, we only selected the first utterance and used a categorical model to predict the probability with which each individual was the speaker of that utterance and thus the initiator of the theme. Once again, we compared a null model including only an overall intercept (\texttt{initiator\ \textasciitilde{}\ 1}) to a model including cultural setting (\texttt{initiator\ \textasciitilde{}\ 1\ +\ (1\textbar{}setting)}).

The model comparison favored the model including cultural setting (WAIC = 6566.90, se = 26.07, weight = 0.73) over the null model (WAIC = 6568.84, se = 25.58, weight = 0.27). However, the difference between models was rather small, suggesting that there were no pronounced differences between cultural settings (dWAIC = -0.97, SE(dWAIC) = 1.61). Overall, there were no huge differences between the three individuals in terms of the probability of being the initiator of a theme (range: 0.26 to 0.41). Compared to an equal split, mothers were slightly more likely to initiate themes and fathers less likely. This relative pattern held for all cultural settings, except Brazil, where the child was the most likely initiator of a theme (see Figure \ref{fig:fig3}B).

\hypertarget{how-long-do-themes-last}{%
\subsection{How long do themes last?}\label{how-long-do-themes-last}}

We finished the analysis of themes by asking about variation in how long themes lasted (i.e., how many utterances there were in a theme). For each theme, we noted its length (i.e., the maximum utterance) and the main interlocutors. For that, we counted how many utterances were exchanged between all possible pairs in each theme and classified each theme as being mainly a conversation between those interlocutors who exchanged the most utterances. As a consequence, we excluded all themes that only had a single utterance and only involved a single speaker. The dependent variable (length of the theme) was heavily right-skewed and close to zero and we, therefore, used a log-normal distribution to model it.

The null model only included an intercept and a random effect of family (\texttt{theme\_length\ \textasciitilde{}\ 1\ +\ (1\textbar{}family)}). There were three alternative models: the first included interlocutors as a fixed effect (\texttt{theme\_length\ \textasciitilde{}\ interlocutors\ +\ (1\textbar{}family)}), the second model added to this a random effect for setting (\texttt{theme\_length\ \textasciitilde{}\ interlocutors\ +\ (1\textbar{}setting)\ +\ (1\textbar{}family)}) and the third model included an additional random slope for interlocutors within setting to model the interaction between interlocutors and setting (\texttt{theme\_length\ \textasciitilde{}\ interlocutors\ +\ (interlocutors\textbar{}setting)\ +\ (1\textbar{}family)}).

The model comparison favored the interaction model assuming that the difference in length of themes for each pair of interlocutors varied across cultural settings (WAIC = 11657.48, se = 106.30, weight \textgreater{} 0.99; second best model (without interaction): WAIC = 11671.33, se = 106.59, weight \textless{} 0.01; dWAIC = -6.92, SE(dWAIC) = 4.11). The average predicted length of a theme across interlocutors and settings was 5.71 utterances {[}95\%CrI = 3.95 - 8.35{]}. Figure \ref{fig:fig3}C indicates a variable pattern across cultural settings. In Japan, themes were approximately equally long for all pairs of interlocutors. In the other settings, conversations between mother and father were shorter compared to conversations between one of the parents and the child. This pattern was less pronounced in Ecuador compared to Germany, Brazil and Argentina. Overall, themes lasted slightly longer in Brazil compared to the other settings.

\begin{figure}
\includegraphics[width=1\linewidth]{../visuals/fig3} \caption{A: Average number of people involved in a theme. B: Proportion of themes as a function of who initiated them. Color shows the initiator. C: Number of utterances per theme depending on the interlocutors involved. Color shows the interlocutors who exchanged the most utterances within a given theme. Distributions show the predicted values based on the respective model with solid points and error bars showing the mean with 66\% and 95\% CrI. Light points show the aggregated data for each family.}\label{fig:fig3}
\end{figure}

\hypertarget{family-level-clustering}{%
\subsection{Family level clustering}\label{family-level-clustering}}

In this final analysis, we took a more holistic look at the data and tried to identify patterns across the communicative dimensions analyzed above. That is, we asked if there were clusters within our sample that represent different communicative profiles. This allowed us to see a) if families clustered based on cultural settings and b) how the different cultural settings clustered with each other. To construct the data set for this analysis, we computed the following dimensions for each family: the amount of \texttt{Non-talk}, the proportion of utterances coming from each individual (\texttt{Father\ speaker}, \texttt{Mother\ speaker}, and \texttt{Child\ speaker}), the proportion of \texttt{Questions}, \texttt{Assertions}, and \texttt{Imperatives}, the number of \texttt{Gestures}, the number of \texttt{Themes}, the average number of \texttt{Utterances} per theme, and the average number of \texttt{Speakers\ per\ theme}. Please note that more granular dimensions (e.g., gestures or functional elements separate for each individual) would have been possible. However, because this would have meant that each dimension would have had to be estimated based on less data (resulting in a more noisy estimate), we decided to use a more coarse approach.

We performed \emph{k}-means clustering on the data using the function \texttt{kmeans} from the \texttt{stats} package which is a native component of \texttt{R}. This analysis partitions the data into \emph{k} clusters so that the sum of squares from points to the assigned cluster centers -- in the multidimensional space that is defined by the different dimensions -- is minimized. We used the default \emph{Hartigan-Wong} algorithm to find these cluster centers (Hartigan \& Wong, 1979). To determine the number of clusters, we used the \emph{silhouette} and \emph{elbow} methods via the function \texttt{fviz\_nbclust} from the \texttt{factoextra} package (Kassambara \& Mundt, 2020). Both suggested two clusters as the optimal solution.

Figure \ref{fig:fig4}A visualizes the clustering of families based on this analysis. The first cluster (blue), included mainly families from Argentina, Germany and Japan. Within the cluster, there was no further clustering of families by cultural setting. The second cluster (gold), mainly comprised families from Ecuador and Brazil. Within that cluster, families further tended to cluster by cultural setting, with families from Brazil being more similar to each other compared to families from Ecuador.

In comparison to the first cluster, the second cluster (mainly Ecuador and Brazil) was characterized by overall less talk, a higher proportion of child- compared to parental-talk, and fewer gestures. Furthermore, there were fewer themes, but themes had more speakers and lasted longer. Finally, there was a higher proportion of imperatives and thus fewer assertions and questions (see Figure \ref{fig:fig4}B).

Figure \ref{fig:fig4}C shows the correlations between the different dimensions across clusters. Besides some expected patterns (e.g., negative correlation between proportion of talk from the different individuals) there were some notable associations: more non-talk was associated with a higher proportion of imperatives, themes had more utterances the more speakers were involved, and a larger number of questions was associated with more themes.

\begin{figure}
\includegraphics[width=1\linewidth]{../visuals/fig4} \caption{A: Dendrogram visualizing the similarity between families based on a cluster analysis assuming two clusters. Line colors show the two clusters, color of letters for family corresponds to the different cultural settings The first letter of the family name denotes the cultural setting (e.g., J = Japan). B: Mean values for the two clusters for each (standardized) dimension on which the cluster analysis was based. C: Pearson correlations between the different dimensions entering the cluster analysis. Color of cells shows the size and direction of the correlation coefficient. Cells without circles show correlations with p-values < 0.05.}\label{fig:fig4}
\end{figure}

\hypertarget{discussion}{%
\section{Discussion}\label{discussion}}

We investigated parent-child communicative interactions during mealtimes in five cultural settings. Each family comprised a father, mother and one child and we analyzed 10 minutes of video recordings. We found that families from Ecuador and Brazil spent less time talking and used fewer gestures compared to families from Argentina and Germany, with Japan falling in the middle. Across settings, there was a common pattern in how talk was distributed across family members: mothers talked the most and children were addressed most frequently. Assertions were the most common type of functional element for all speakers in all settings, followed by questions and imperatives. However, mothers and fathers form Brazil and Ecuador were more likely to use imperatives. The number of themes -- parts of coherent utterances -- tended to be longer and involved more people in Brazil compared to the other settings. When investigating how families clustered based on their communicative interaction patterns, we found what can be described as an urban-rural split, with families from urban settings (Argentina, Germany, Japan) being more similar to each other compared to families from rural settings (Brazil, Ecuador). These systematic, quantitative comparisons provide an important step towards understanding the similarities and differences in communicative contexts in which children learn language.

Our findings echo how Barrett (2020; see also Kärtner, Schuhmacher, \& Giner Torréns, 2020) summarized much of cross-cultural research in the last two decades: \emph{variation on a theme}. For every aspect of communicative interaction we investigated, there was a dominant pattern which described behavior in most of the cultural settings, but which was often modified in one or two settings. Modification meant that the predicted means for some of the settings were shifted while the distributions of families were largely overlapping. For example, on average, the number of people involved in a theme was around 1.8, with the highest predicted average for Brazil (\textasciitilde{} 2.1) and the lowest for Ecuador (\textasciitilde1.6), yet, the minimum family average in Brazil was 1.40 and the maximum for Ecuador was 2. Similarly, mothers talked the most in all settings but the difference compared to fathers and children was less pronounced in Ecuador and Brazil. Thus, we may tentatively conclude that these overlaps in communicative patterns allow children to use similar learning strategies across settings -- in particular those strategies that leverage the structure of the communicative context (Casillas et al., 2020; Rogoff, Paradise, et al., 2003; Shneidman \& Goldin-Meadow, 2012; Shneidman \& Woodward, 2016).

The overall pattern -- or \emph{theme} -- can be summarized as being child-centered. Across cultural settings, most talk was directed towards the child. This lends support to theories highlighting the role of direct input for language learning (Braginsky et al., 2019; Goodman et al., 2008; Kachergis et al., 2022; Swingley \& Humphrey, 2018). Despite absolute differences in how much input children received, across settings parents directed the largest proportion of talk at the child. Meals are structured by cultural norms which the child has yet to learn, resulting in more direct instruction and -- as a by-product -- more child-directed linguistic input (Blum-Kulka, 2012; Fjellström, 2004; Köster et al., 2022; Ochs \& Shohet, 2006). Furthermore, themes had more conversational turns (i.e., number of utterances) when the child was involved. This finding corresponds well with the idea that children's language learning benefits from coherent and structured interactions (Casillas et al., 2020; Rogoff, Paradise, et al., 2003; Shneidman \& Goldin-Meadow, 2012; Shneidman \& Woodward, 2016). More frequent conversational turns could originate from adults gradually adjusting and elaborating their utterances to the child's response (or lack thereof), resulting in a form of linguistic scaffolding (Bruner, 1983; Vygotsky, 1978). Taken together, the child-centered way of communication might be the consequence of how the interactions in which talk occurs are structured.

Mothers seemed to be the driving force behind this child-centered communicative pattern: they spoke the most, initiated most themes and most of the themes they were involved in also included the child. This aligns with the former analyses of these videos showing that mothers teach more compared to fathers (Köster et al., 2022) and a recent study by Broesch et al. (2021) who described mothers as the primary interaction partners for young children across five cultural settings. Fathers spoke less and were less likely to be involved in a conversation with the child. As mentioned above, this overall pattern was modified in some of the cultural settings and in the following we will take a closer look at this variation.

The cluster analyses showed that families' communicative interaction patterns co-varied with the degree of urbanization. Families from Brazil and Ecuador were more similar to each other than they were to families from Argentina, Germany and Japan. Interestingly, within the rural cluster, there seemed to be a further grouping by setting. This was not the case within the urban cluster: even though they lived in very different geographical regions and spoke very different languages. That is, families from Argentina, Germany and Japan were not more similar to families from the same setting than they were to families from the other settings. However, the urban/rural split was by no means complete in that some of the families from Brazil and Ecuador were assigned to the urban cluster and some families from Argentina were grouped in the rural cluster. A similar difference between urban and rural settings was found when analyzing parental teaching behavior for these samples but with a stronger sub-clustering of families in the urban cluster (Köster et al., 2022). Taken together, these results show that variation in communicative interactions did not -- at least not primarily -- originate from the languages that were spoken, but might have been due to norms, values and beliefs prevalent in the respective cultural settings.

Several theoretical frameworks have focused on different parental socialization goals in urban and rural settings. For example, Keller (2007) described that parents in urban settings prioritize children's independence while parents in rural settings prioritize interdependence. In line with these proposals, we found that parents in urban samples used more questions, and parents in rural samples used more imperatives, likely reflecting an emphasis on autonomy and compliance, respectively. Furthermore, themes included more speakers in the rural contexts which could reflect a stronger orientation towards the group as opposed to the dyad (Rogoff et al., 2003). Not in line with this general interpretation was the finding that children spoke more in the rural context. Children in urban settings have been described as more communicative because they receive more prompts from their caregivers (Keller, 2007). Below we discuss in more detail how specific norms, values and beliefs may have influenced the communicative interactions.

Families from Brazil and Ecuador had longer periods of non-talk and produced fewer gestures compared to families from Argentina and Germany. Japanese families fell somewhere in between. This mirrors results by Cristia (2022) who synthesized 29 studies on naturalistic language input and found that infants growing up in rural settings heard less child-directed speech compared to children growing up in urban settings. It is also in line with the cultural norms that have been described for some of the settings. For the Kichwa community in Ecuador, Sánchez-Parga (2010) reports a norm that meals are supposed to be taken in silence. In Japan, meals are also supposed to be silent under some circumstances (Imada \& Furumitsu, 2020). In Germany and Argentina, family meals are seen as a privileged occasion for communication (Aguirre, 2016; Danesi, 2018). In our sample, such norms seemed to have influenced mothers' communication the most: there was less talk by mothers in Ecuador compared to the other settings (except Brazil), while the amount of talk by fathers and children was relatively similar. However, given that all family members talked in all settings, it is worth pointing out that such norms -- at least in the present study -- mainly had an attenuating effect.

Children communicated in very similar ways across settings: they mostly made assertions and rarely asked questions or used imperatives. Parents' communication in the different settings were also very similar in that they mostly made assertions, asked relatively few questions and hardly used any imperatives. Notably, the rate of imperatives was substantially higher in the rural settings in Brazil and Ecuador. For rural Brazil, Köster and colleagues (2016) reported that mothers assigned tasks to their children in a more assertive way compared to mothers from urban Germany (see also Keller et al., 2004 for similar findings from rural Costa Rica). Furthermore, when Köster et al. (2022) coded teaching behavior in the same samples, they found that a higher rate of parents in Brazil and Ecuador prompted their children to do something. Finally, in a study on norm enforcement, children living in rural settings themselves used more imperatives than norm-protest when reacting to a peer's perceived norm violation (Kanngiesser et al., 2022). Thus, the higher rate of imperatives might reflect cultural norms and beliefs about how children should behave and how they learn (Keller, 2007).

\hypertarget{limitations}{%
\section{Limitations}\label{limitations}}

We see the mealtime setting in which we investigated communicative interactions among family members as a strength of the current study, but acknowledge that it comes with important limitations. The constellation of mother, father and one child is probably more representative for the urban contexts of Argentina, Germany and Japan than the rural settings. Thus, it would be interesting to see if and how our observed patterns are changed when more people (especially more children and extended family members) take part in the meal. Based on our current findings, we would anticipate similar rates of change across cultural settings. For example, we would expect that the presence of a second child would lower the rate of talk addressed to the other child in a similar way in all cultural settings. Of course, this prediction -- as well as all our results -- can only generalize to cultural settings in which the interaction format of joint mealtimes exists.

Furthermore, our sample was a convenience sample in that we relied on established contacts and collaborations to recruit families in different settings. As such, the grouping into rural and urban contexts is confounded with the normative belief systems of particular regions. Thus, we do not think that living in a rural setting per se affects communicative interactions in a systematic way but the specific cultural norms and practices associated with rural subsistence in these settings produced the patterns we observed. Future work should combine our quantitative approach with a qualitative assessment of the local norms surrounding communication and mealtime to better understand the link between norms, values and beliefs and communicative behavior.

Finally, we did not obtain a measure of children's language abilities. As such, we can only speculate to what extent the different interaction patterns directly affected children's language learning. Obtaining such measures would be a valuable extension of our work.

\hypertarget{conclusions}{%
\section{Conclusions}\label{conclusions}}

Our findings offer important insights into the variable and constant aspects of children's language learning environments across diverse cultural settings. For all aspects of communication we investigated in the current study, a common pattern emerged across cultural settings suggesting that children can rely on similar information sources and learning processes. This common pattern was modified in some of the settings in a way that might reflect particular local norms, values, beliefs and ecologies. This exemplifies the importance of quantitative cross-cultural research for theory building in language acquisition.

\newpage

\hypertarget{references}{%
\section{References}\label{references}}

\hypertarget{refs}{}
\begin{CSLReferences}{1}{0}
\leavevmode\vadjust pre{\hypertarget{ref-aguirre2016olla}{}}%
Aguirre, P. (2016). La olla, la fuente y el plato. Distintas maneras de compartir en argentina: The pot, the platter and the dish: Various ways of sharing in argentina. \emph{Studium}, (22), 189--208.

\leavevmode\vadjust pre{\hypertarget{ref-akhtar2002relevance}{}}%
Akhtar, N. (2002). Relevance and early word learning. \emph{Journal of Child Language}, \emph{29}(3), 677--686.

\leavevmode\vadjust pre{\hypertarget{ref-altinkamics2014context}{}}%
Altınkamış, N. F., Kern, S., \& Sofu, H. (2014). When context matters more than language: Verb or noun in french and turkish caregiver speech. \emph{First Language}, \emph{34}(6), 537--550.

\leavevmode\vadjust pre{\hypertarget{ref-amir2020cross}{}}%
Amir, D., \& McAuliffe, K. (2020). Cross-cultural, developmental psychology: Integrating approaches and key insights. \emph{Evolution and Human Behavior}, \emph{41}(5), 430--444.

\leavevmode\vadjust pre{\hypertarget{ref-bang2022spanish}{}}%
Bang, J. Y., Bohn, M., Ramirez, J., Marchman, V. A., \& Fernald, A. (2022). Spanish-speaking caregivers' use of referential labels with toddlers is a better predictor of later vocabulary than their use of referential gestures. \emph{PsyArXiv}.

\leavevmode\vadjust pre{\hypertarget{ref-bang2022time}{}}%
Bang, J. Y., Mora, A., Munévar, M., Fernald, A., \& Marchman, V. A. (2022). \emph{Time to talk: Multiple sources of variability in caregiver verbal engagement during everyday activities in english-and spanish-speaking families in the US}.

\leavevmode\vadjust pre{\hypertarget{ref-de2022ten}{}}%
Barbaro, K. de, \& Fausey, C. M. (2022). Ten lessons about infants' everyday experiences. \emph{Current Directions in Psychological Science}, \emph{31}(1), 28--33.

\leavevmode\vadjust pre{\hypertarget{ref-barrett2020towards}{}}%
Barrett, H. C. (2020). Towards a cognitive science of the human: Cross-cultural approaches and their urgency. \emph{Trends in Cognitive Sciences}, \emph{24}(8), 620--638.

\leavevmode\vadjust pre{\hypertarget{ref-beals1993explanatory}{}}%
Beals, D. E. (1993). Explanatory talk in low-income families' mealtime conversations. \emph{Applied Psycholinguistics}, \emph{14}(4), 489--513.

\leavevmode\vadjust pre{\hypertarget{ref-beals1997sources}{}}%
Beals, D. E. (1997). Sources of support for learning words in conversation: Evidence from mealtimes. \emph{Journal of Child Language}, \emph{24}(3), 673--694.

\leavevmode\vadjust pre{\hypertarget{ref-bergelson2019north}{}}%
Bergelson, E., Casillas, M., Soderstrom, M., Seidl, A., Warlaumont, A. S., \& Amatuni, A. (2019). What do north american babies hear? A large-scale cross-corpus analysis. \emph{Developmental Science}, \emph{22}(1), e12724.

\leavevmode\vadjust pre{\hypertarget{ref-blake2006book}{}}%
Blake, J., Macdonald, S., Bayrami, L., Agosta, V., \& Milian, A. (2006). Book reading styles in dual-parent and single-mother families. \emph{British Journal of Educational Psychology}, \emph{76}(3), 501--515.

\leavevmode\vadjust pre{\hypertarget{ref-blake2015ontogeny}{}}%
Blake, P. R., McAuliffe, K., Corbit, J., Callaghan, T. C., Barry, O., Bowie, A., et al.others. (2015). The ontogeny of fairness in seven societies. \emph{Nature}, \emph{528}(7581), 258--261.

\leavevmode\vadjust pre{\hypertarget{ref-blum2012dinner}{}}%
Blum-Kulka, S. (2012). \emph{Dinner talk: Cultural patterns of sociability and socialization in family discourse}. Routledge.

\leavevmode\vadjust pre{\hypertarget{ref-bohn2019pervasive}{}}%
Bohn, M., \& Frank, M. C. (2019). The pervasive role of pragmatics in early language. \emph{Annual Review of Developmental Psychology}, \emph{1}(1), 223--249.

\leavevmode\vadjust pre{\hypertarget{ref-bohn2018common}{}}%
Bohn, M., \& Köymen, B. (2018). Common ground and development. \emph{Child Development Perspectives}, \emph{12}(2), 104--108.

\leavevmode\vadjust pre{\hypertarget{ref-bohn_le_peloquin_koymen_frank_2020}{}}%
Bohn, M., Le, K. N., Peloquin, B., Köymen, B., \& Frank, M. C. (2021). Children's interpretation of ambiguous pronouns based on prior discourse. \emph{Developmental Science}, \emph{24}(3), e13049.

\leavevmode\vadjust pre{\hypertarget{ref-bohn2021young}{}}%
Bohn, M., Tessler, M. H., Merrick, M., \& Frank, M. C. (2021). How young children integrate information sources to infer the meaning of words. \emph{Nature Human Behaviour}, \emph{5}(8), 1046--1054.

\leavevmode\vadjust pre{\hypertarget{ref-braginsky2019consistency}{}}%
Braginsky, M., Yurovsky, D., Marchman, V. A., \& Frank, M. C. (2019). Consistency and variability in children's word learning across languages. \emph{Open Mind}, \emph{3}, 52--67.

\leavevmode\vadjust pre{\hypertarget{ref-broesch2021opportunities}{}}%
Broesch, T., Carolan, P. L., Cebioğlu, S., Rueden, C. von, Boyette, A., Moya, C., \ldots{} Kline, M. A. (2021). Opportunities for interaction: Natural observations of children's social behavior in five societies. \emph{Human Nature}.

\leavevmode\vadjust pre{\hypertarget{ref-broesch2022roadmap}{}}%
Broesch, T., Lew-Levy, S., Kärtner, J., Kanngiesser, P., \& Kline, M. (2022). A roadmap to doing culturally grounded developmental science. \emph{Review of Philosophy and Psychology}, 1--23.

\leavevmode\vadjust pre{\hypertarget{ref-brown2014language}{}}%
Brown, P., \& Gaskins, S. (2014). Language acquisition and language socialization. In N. J. Enfield, P. Kockelman, \& J. Sidnell (Eds.), \emph{Cambridge handbook of linguistic anthropology} (pp. 187--226). Cambridge University Press.

\leavevmode\vadjust pre{\hypertarget{ref-bruner1983child}{}}%
Bruner, J. (1983). \emph{Child's talk: Learning to use language}. New York: Norton.

\leavevmode\vadjust pre{\hypertarget{ref-bunce2020cross}{}}%
Bunce, J., Soderstrom, M., Bergelson, E., Rosemberg, C., Stein, A., Migdalek, M., et al.others. (2020). \emph{A cross-cultural examination of young children's everyday language experiences}.

\leavevmode\vadjust pre{\hypertarget{ref-burkner2017brms}{}}%
Bürkner, P.-C. (2017). Brms: An r package for bayesian multilevel models using stan. \emph{Journal of Statistical Software}, \emph{80}(1), 1--28.

\leavevmode\vadjust pre{\hypertarget{ref-casillas2022learning}{}}%
Casillas, M. (2022). Learning language in vivo. \emph{Child Development Perspectives}.

\leavevmode\vadjust pre{\hypertarget{ref-casillas2020early}{}}%
Casillas, M., Brown, P., \& Levinson, S. C. (2020). Early language experience in a tseltal mayan village. \emph{Child Development}, \emph{91}(5), 1819--1835.

\leavevmode\vadjust pre{\hypertarget{ref-casillas2021early}{}}%
Casillas, M., Brown, P., \& Levinson, S. C. (2021). Early language experience in a papuan community. \emph{Journal of Child Language}, \emph{48}(4), 792--814.

\leavevmode\vadjust pre{\hypertarget{ref-choi2000caregiver}{}}%
Choi, S. (2000). Caregiver input in english and korean: Use of nouns and verbs in book-reading and toy-play contexts. \emph{Journal of Child Language}, \emph{27}(1), 69--96.

\leavevmode\vadjust pre{\hypertarget{ref-clark2009first}{}}%
Clark, E. V. (2009). \emph{First language acquisition}. Cambridge: Cambridge University Press.

\leavevmode\vadjust pre{\hypertarget{ref-colonnesi2010relation}{}}%
Colonnesi, C., Stams, G. J. J., Koster, I., \& Noom, M. J. (2010). The relation between pointing and language development: A meta-analysis. \emph{Developmental Review}, \emph{30}(4), 352--366.

\leavevmode\vadjust pre{\hypertarget{ref-cristia2022systematic}{}}%
Cristia, A. (2022). A systematic review suggests marked differences in the prevalence of infant-directed vocalization across groups of populations. \emph{Developmental Science}, e13265.

\leavevmode\vadjust pre{\hypertarget{ref-cristia2019child}{}}%
Cristia, A., Dupoux, E., Gurven, M., \& Stieglitz, J. (2019). Child-directed speech is infrequent in a forager-farmer population: A time allocation study. \emph{Child Development}, \emph{90}(3), 759--773.

\leavevmode\vadjust pre{\hypertarget{ref-cristia2021thorough}{}}%
Cristia, A., Lavechin, M., Scaff, C., Soderstrom, M., Rowland, C., Räsänen, O., \ldots{} Bergelson, E. (2021). A thorough evaluation of the language environment analysis (LENA) system. \emph{Behavior Research Methods}, \emph{53}(2), 467--486.

\leavevmode\vadjust pre{\hypertarget{ref-danesi2018cross}{}}%
Danesi, G. (2018). A cross-cultural approach to eating together: Practices of commensality among french, german and spanish young adults. \emph{Social Science Information}, \emph{57}(1), 99--120.

\leavevmode\vadjust pre{\hypertarget{ref-de2011language}{}}%
De León, L. (2011). Language socialization and multiparty participation frameworks. In A. Duranti, E. Ochs, \& B. B. Schieffelin (Eds.), \emph{The handbook of language socialization} (pp. 81--111). Malden, MA: Wiley-Blackwell.

\leavevmode\vadjust pre{\hypertarget{ref-donnellan2020infants}{}}%
Donnellan, E., Bannard, C., McGillion, M. L., Slocombe, K. E., \& Matthews, D. (2020). Infants' intentionally communicative vocalizations elicit responses from caregivers and are the best predictors of the transition to language: A longitudinal investigation of infants' vocalizations, gestures and word production. \emph{Developmental Science}, \emph{23}(1), e12843.

\leavevmode\vadjust pre{\hypertarget{ref-donnelly2021longitudinal}{}}%
Donnelly, S., \& Kidd, E. (2021). The longitudinal relationship between conversational turn-taking and vocabulary growth in early language development. \emph{Child Development}, \emph{92}(2), 609--625.

\leavevmode\vadjust pre{\hypertarget{ref-ferjan2022fathers}{}}%
Ferjan Ramírez, N. (2022). Fathers' infant-directed speech and its effects on child language development. \emph{Language and Linguistics Compass}, \emph{16}(1), e12448.

\leavevmode\vadjust pre{\hypertarget{ref-fjellstrom2004mealtime}{}}%
Fjellström, C. (2004). Mealtime and meal patterns from a cultural perspective. \emph{Scandinavian Journal of Nutrition}, \emph{48}(4), 161--164.

\leavevmode\vadjust pre{\hypertarget{ref-gaskins2006cultural}{}}%
Gaskins, S. (2006). Cultural perspectives on InfantCaregiver interaction. In N. J. Enfield \& S. Levinson (Eds.), \emph{Roots of human sociality: Culture, cognition and interaction} (pp. 279--298). Oxford, UK: Berg.

\leavevmode\vadjust pre{\hypertarget{ref-gelman2006data}{}}%
Gelman, A., \& Hill, J. (2006). \emph{Data analysis using regression and multilevel/hierarchical models}. Cambridge university press.

\leavevmode\vadjust pre{\hypertarget{ref-goodman2008does}{}}%
Goodman, J. C., Dale, P. S., \& Li, P. (2008). Does frequency count? Parental input and the acquisition of vocabulary. \emph{Journal of Child Language}, \emph{35}(3), 515--531.

\leavevmode\vadjust pre{\hypertarget{ref-greenwood2011assessing}{}}%
Greenwood, C. R., Thiemann-Bourque, K., Walker, D., Buzhardt, J., \& Gilkerson, J. (2011). Assessing children's home language environments using automatic speech recognition technology. \emph{Communication Disorders Quarterly}, \emph{32}(2), 83--92.

\leavevmode\vadjust pre{\hypertarget{ref-hart1995meaningful}{}}%
Hart, B., \& Risley, T. R. (1995). \emph{Meaningful differences in the everyday experience of young american children.} Paul H Brookes Publishing.

\leavevmode\vadjust pre{\hypertarget{ref-hartigan1979algorithm}{}}%
Hartigan, J. A., \& Wong, M. A. (1979). Algorithm AS 136: A k-means clustering algorithm. \emph{Journal of the Royal Statistical Society. Series C (Applied Statistics)}, \emph{28}(1), 100--108.

\leavevmode\vadjust pre{\hypertarget{ref-hewlett1998culture}{}}%
Hewlett, B. S., Lamb, M. E., Shannon, D., Leyendecker, B., \& Schölmerich, A. (1998). Culture and early infancy among central african foragers and farmers. \emph{Developmental Psychology}, \emph{34}(4), 653.

\leavevmode\vadjust pre{\hypertarget{ref-house2020universal}{}}%
House, B. R., Kanngiesser, P., Barrett, H. C., Broesch, T., Cebioglu, S., Crittenden, A. N., et al.others. (2020). Universal norm psychology leads to societal diversity in prosocial behaviour and development. \emph{Nature Human Behaviour}, \emph{4}(1), 36--44.

\leavevmode\vadjust pre{\hypertarget{ref-house2013ontogeny}{}}%
House, B. R., Silk, J. B., Henrich, J., Barrett, H. C., Scelza, B. A., Boyette, A. H., \ldots{} Laurence, S. (2013). Ontogeny of prosocial behavior across diverse societies. \emph{Proceedings of the National Academy of Sciences}, \emph{110}(36), 14586--14591.

\leavevmode\vadjust pre{\hypertarget{ref-huttenlocher1991early}{}}%
Huttenlocher, J., Haight, W., Bryk, A., Seltzer, M., \& Lyons, T. (1991). Early vocabulary growth: Relation to language input and gender. \emph{Developmental Psychology}, \emph{27}(2), 236.

\leavevmode\vadjust pre{\hypertarget{ref-imada2020traditional}{}}%
Imada, S., \& Furumitsu, I. (2020). Traditional and modern eating in japan. \emph{Handbook of Eating and Drinking: Interdisciplinary Perspectives}, 1343--1366.

\leavevmode\vadjust pre{\hypertarget{ref-jones2017diversity}{}}%
Jones, G., \& Rowland, C. F. (2017). Diversity not quantity in caregiver speech: Using computational modeling to isolate the effects of the quantity and the diversity of the input on vocabulary growth. \emph{Cognitive Psychology}, \emph{98}, 1--21.

\leavevmode\vadjust pre{\hypertarget{ref-kachergis2022toward}{}}%
Kachergis, G., Marchman, V. A., \& Frank, M. C. (2022). Toward a {``standard model''} of early language learning. \emph{Current Directions in Psychological Science}, \emph{31}(1), 20--27.

\leavevmode\vadjust pre{\hypertarget{ref-kanngiesser2022children}{}}%
Kanngiesser, P., Schäfer, M., Herrmann, E., Zeidler, H., Haun, D., \& Tomasello, M. (2022). Children across societies enforce conventional norms but in culturally variable ways. \emph{Proceedings of the National Academy of Sciences}, \emph{119}(1), e2112521118.

\leavevmode\vadjust pre{\hypertarget{ref-kartner2020culture}{}}%
Kärtner, J., Schuhmacher, N., \& Giner Torréns, M. (2020). Culture and early social-cognitive development. \emph{Progress in Brain Research}, \emph{254}, 225--246.

\leavevmode\vadjust pre{\hypertarget{ref-factoextract}{}}%
Kassambara, A., \& Mundt, F. (2020). \emph{Factoextra: Extract and visualize the results of multivariate data analyses}. Retrieved from \url{https://CRAN.R-project.org/package=factoextra}

\leavevmode\vadjust pre{\hypertarget{ref-keller2007cultures}{}}%
Keller, H. (2007). \emph{Cultures of infancy}. Lawrence Erlbaum Associates.

\leavevmode\vadjust pre{\hypertarget{ref-keller2004developmental}{}}%
Keller, H., Yovsi, R., Borke, J., Kärtner, J., Jensen, H., \& Papaligoura, Z. (2004). Developmental consequences of early parenting experiences: Self-recognition and self-regulation in three cultural communities. \emph{Child Development}, \emph{75}(6), 1745--1760.

\leavevmode\vadjust pre{\hypertarget{ref-kidd2022diverse}{}}%
Kidd, E., \& Garcia, R. (2022). How diverse is child language acquisition research? \emph{First Language}, 01427237211066405.

\leavevmode\vadjust pre{\hypertarget{ref-koster2016cultural}{}}%
Köster, M., Cavalcante, L., Vera Cruz de Carvalho, R., Dôgo Resende, B., \& Kärtner, J. (2016). Cultural influences on toddlers' prosocial behavior: How maternal task assignment relates to helping others. \emph{Child Development}, \emph{87}(6), 1727--1738.

\leavevmode\vadjust pre{\hypertarget{ref-koester2022parental}{}}%
Köster, M., Giner Torréns, M., Kärtner, J., Itakura, S., Cavalcante, L., \& Kanngiesser, P. (2022). Parental teaching behavior in diverse cultural contexts. \emph{Evolution and Human Behavior}.

\leavevmode\vadjust pre{\hypertarget{ref-kuchirko2020say}{}}%
Kuchirko, Y. A., Schatz, J. L., Fletcher, K. K., \& Tamis-Lemonda, C. S. (2020). Do, say, learn: The functions of mothers' speech to infants. \emph{Journal of Child Language}, \emph{47}(1), 64--84.

\leavevmode\vadjust pre{\hypertarget{ref-lavechin2020open}{}}%
Lavechin, M., Bousbib, R., Bredin, H., Dupoux, E., \& Cristia, A. (2020). An open-source voice type classifier for child-centered daylong recordings. \emph{arXiv Preprint arXiv:2005.12656}.

\leavevmode\vadjust pre{\hypertarget{ref-levinson2014origin}{}}%
Levinson, S. C., \& Holler, J. (2014). The origin of human multi-modal communication. \emph{Philosophical Transactions of the Royal Society B: Biological Sciences}, \emph{369}(1651), 20130302.

\leavevmode\vadjust pre{\hypertarget{ref-lieven1994crosslinguistic}{}}%
Lieven, E. (1994). Crosslinguistic and crosscultural aspects of language addressed to children. In C. Gallaway \& B. J. Richards (Eds.), \emph{Input and interaction in language acquisition} (pp. 56--73). New York: Cambridge University Press.

\leavevmode\vadjust pre{\hypertarget{ref-loukatou2021child}{}}%
Loukatou, G., Scaff, C., Demuth, K., Cristia, A., \& Havron, N. (2021). Child-directed and overheard input from different speakers in two distinct cultures. \emph{Journal of Child Language}, 1--20.

\leavevmode\vadjust pre{\hypertarget{ref-masek2021beyond}{}}%
Masek, L. R., Ramirez, A. G., McMillan, B. T., Hirsh-Pasek, K., \& Golinkoff, R. M. (2021). Beyond counting words: A paradigm shift for the study of language acquisition. \emph{Child Development Perspectives}, \emph{15}(4), 274--280.

\leavevmode\vadjust pre{\hypertarget{ref-mcelreath2018statistical}{}}%
McElreath, R. (2018). \emph{Statistical rethinking: A bayesian course with examples in r and stan}. Chapman; Hall/CRC.

\leavevmode\vadjust pre{\hypertarget{ref-mcmurray2012word}{}}%
McMurray, B., Horst, J. S., \& Samuelson, L. K. (2012). Word learning emerges from the interaction of online referent selection and slow associative learning. \emph{Psychological Review}, \emph{119}(4), 831.

\leavevmode\vadjust pre{\hypertarget{ref-meylan2022learning}{}}%
Meylan, S. C., \& Bergelson, E. (2022). Learning through processing: Toward an integrated approach to early word learning. \emph{Annual Review of Linguistics}, \emph{8}(1), 77--99. \url{https://doi.org/10.1146/annurev-linguistics-031220-011146}

\leavevmode\vadjust pre{\hypertarget{ref-nielsen2017persistent}{}}%
Nielsen, M., Haun, D., Kärtner, J., \& Legare, C. H. (2017). The persistent sampling bias in developmental psychology: A call to action. \emph{Journal of Experimental Child Psychology}, \emph{162}, 31--38.

\leavevmode\vadjust pre{\hypertarget{ref-ochs2006cultural}{}}%
Ochs, E., \& Shohet, M. (2006). The cultural structuring of mealtime socialization. \emph{New Directions for Child and Adolescent Development}, \emph{2006}(111), 35--49.

\leavevmode\vadjust pre{\hypertarget{ref-citeR}{}}%
R Core Team. (2022). \emph{R: A language and environment for statistical computing}. Vienna, Austria: R Foundation for Statistical Computing. Retrieved from \url{https://www.R-project.org/}

\leavevmode\vadjust pre{\hypertarget{ref-rogoff2003cultural}{}}%
Rogoff, B. et al. (2003). \emph{The cultural nature of human development}. Oxford University Press.

\leavevmode\vadjust pre{\hypertarget{ref-rogoff2003firsthand}{}}%
Rogoff, B., Paradise, R., Arauz, R. M., Correa-Chávez, M., \& Angelillo, C. (2003). Firsthand learning through intent participation. \emph{Annual Review of Psychology}, \emph{54}(1), 175--203.

\leavevmode\vadjust pre{\hypertarget{ref-romeo2018beyond}{}}%
Romeo, R. R., Leonard, J. A., Robinson, S. T., West, M. R., Mackey, A. P., Rowe, M. L., \& Gabrieli, J. D. (2018). Beyond the 30-million-word gap: Children's conversational exposure is associated with language-related brain function. \emph{Psychological Science}, \emph{29}(5), 700--710.

\leavevmode\vadjust pre{\hypertarget{ref-rowe2008learning}{}}%
Rowe, M. L., Özçalışkan, Ş., \& Goldin-Meadow, S. (2008). Learning words by hand: Gesture's role in predicting vocabulary development. \emph{First Language}, \emph{28}(2), 182--199.

\leavevmode\vadjust pre{\hypertarget{ref-rowe2020language}{}}%
Rowe, M. L., \& Weisleder, A. (2020). Language development in context. \emph{Annual Review of Developmental Psychology}, \emph{2}, 201--223.

\leavevmode\vadjust pre{\hypertarget{ref-roy2015predicting}{}}%
Roy, B. C., Frank, M. C., DeCamp, P., Miller, M., \& Roy, D. (2015). Predicting the birth of a spoken word. \emph{Proceedings of the National Academy of Sciences}, \emph{112}(41), 12663--12668.

\leavevmode\vadjust pre{\hypertarget{ref-sanchez2010oficio}{}}%
Sánchez-Parga, J. (2010). \emph{El oficio de antrop{ó}logo. Cr{í}ticia de la raz{ó}n (inter) cultural.} Ediciones Abya-Yala, Quito.

\leavevmode\vadjust pre{\hypertarget{ref-shneidman2012language}{}}%
Shneidman, L., \& Goldin-Meadow, S. (2012). Language input and acquisition in a mayan village: How important is directed speech? \emph{Developmental Science}, \emph{15}(5), 659--673.

\leavevmode\vadjust pre{\hypertarget{ref-shneidman2016child}{}}%
Shneidman, L., \& Woodward, A. L. (2016). Are child-directed interactions the cradle of social learning? \emph{Psychological Bulletin}, \emph{142}(1), 1.

\leavevmode\vadjust pre{\hypertarget{ref-snow2006mealtime}{}}%
Snow, C. E., \& Beals, D. E. (2006). Mealtime talk that supports literacy development. \emph{New Directions for Child and Adolescent Development}, \emph{2006}(111), 51--66.

\leavevmode\vadjust pre{\hypertarget{ref-soderstrom2013caregivers}{}}%
Soderstrom, M., \& Wittebolle, K. (2013). When do caregivers talk? The influences of activity and time of day on caregiver speech and child vocalizations in two childcare environments. \emph{PloS One}, \emph{8}(11), e80646.

\leavevmode\vadjust pre{\hypertarget{ref-sperry2019reexamining}{}}%
Sperry, D. E., Sperry, L. L., \& Miller, P. J. (2019). Reexamining the verbal environments of children from different socioeconomic backgrounds. \emph{Child Development}, \emph{90}(4), 1303--1318.

\leavevmode\vadjust pre{\hypertarget{ref-swingley2018quantitative}{}}%
Swingley, D., \& Humphrey, C. (2018). Quantitative linguistic predictors of infants' learning of specific english words. \emph{Child Development}, \emph{89}(4), 1247--1267.

\leavevmode\vadjust pre{\hypertarget{ref-tamis2019routine}{}}%
Tamis-LeMonda, C. S., Custode, S., Kuchirko, Y., Escobar, K., \& Lo, T. (2019). Routine language: Speech directed to infants during home activities. \emph{Child Development}, \emph{90}(6), 2135--2152.

\leavevmode\vadjust pre{\hypertarget{ref-tamis2018taking}{}}%
Tamis-LeMonda, C. S., Kuchirko, Y., \& Suh, D. D. (2018). Taking center stage: Infants' active role in language learning. \emph{Active Learning from Infancy to Childhood: Social Motivation, Cognition, and Linguistic Mechanisms}, 39--53.

\leavevmode\vadjust pre{\hypertarget{ref-tardif1997caregiver}{}}%
Tardif, T., Shatz, M., \& Naigles, L. (1997). Caregiver speech and children's use of nouns versus verbs: A comparison of english, italian, and mandarin. \emph{Journal of Child Language}, \emph{24}(3), 535--565.

\leavevmode\vadjust pre{\hypertarget{ref-tomasello2005constructing}{}}%
Tomasello, M. (2005). \emph{Constructing a language: A usage-based theory of language acquisition}. Harvard University Press.

\leavevmode\vadjust pre{\hypertarget{ref-tomasello2009constructing}{}}%
Tomasello, M. (2009). \emph{Constructing a language}. Cambridge, MA: Harvard University Press.

\leavevmode\vadjust pre{\hypertarget{ref-vygotsky1978mind}{}}%
Vygotsky, L. S. (1978). \emph{Mind in society: Development of higher psychological processes}. Harvard University Press.

\leavevmode\vadjust pre{\hypertarget{ref-walker1994prediction}{}}%
Walker, D., Greenwood, C., Hart, B., \& Carta, J. (1994). Prediction of school outcomes based on early language production and socioeconomic factors. \emph{Child Development}, \emph{65}(2), 606--621.

\leavevmode\vadjust pre{\hypertarget{ref-weisleder2013talking}{}}%
Weisleder, A., \& Fernald, A. (2013). Talking to children matters: Early language experience strengthens processing and builds vocabulary. \emph{Psychological Science}, \emph{24}(11), 2143--2152.

\leavevmode\vadjust pre{\hypertarget{ref-wittenburg_elan_2006}{}}%
Wittenburg, P., Brugman, H., Russel, A., Klassman, A., \& Sloetjes, H. (2006). {ELAN}: {A} professional framework for multimodality research. \emph{Proceedings of LREC 2006, Fifth International Conference on Language Resources and Evaluation}. Retrieved from \url{https://archive.mpi.nl/tla/elan}

\end{CSLReferences}


\end{document}
